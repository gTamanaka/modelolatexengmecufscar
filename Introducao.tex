A seguir esse documento dará breve explicações de como usar alguns comandos e montar sua monografia,
 você pode obter ajuda de como configurar ou utilizações mais avançadas 
 \href{http://ctan.sharelatex.com/tex-archive/macros/latex/contrib/abntex2/doc/abntex2.pdf}{aqui} 
 (Dê download no pdf ou olhe o código fonte)

\section{Coisinhas importantes para saber}
	Quando você dá enter em uma linha no Latex isso não gera um novo paragráfo. Você precisa dar
	dois ENTER. 

\section{Como dividir o seu trabalho}
	A ABNT define subdivisões de trabalho até o level 7, as divisões começam no capítulo, 
	passando por seção, subseção e subsub... portanto se você quiser dividir o seu trabalho de acordo 
	você pode usar \textbackslash chapter, \textbackslash section, \textbackslash subsection, \textbackslash subsubsection 
	and keep going ... O sumário vai ficar certinho, não precisa se preocupar.


	Vamos as demonstrações
\section{Seção}
\subsection{Sub}
\subsubsection{SubSub}
\subsubsubsection{SubSubSub}
