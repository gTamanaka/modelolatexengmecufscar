\documentclass[12pt,openright,twoside,a4paper,chapter=TITLE,english,brazil,]{abntex2}
%Pacotes
\usepackage{lmodern}			
\usepackage[T1]{fontenc}		
\usepackage[utf8]{inputenc}		
\usepackage{indentfirst}		
\usepackage{color}				
\usepackage{graphicx}			
\usepackage{subfig}
\usepackage{microtype} 			
\usepackage{lipsum}	
\usepackage{hyperref}
\usepackage[hyphenbreaks]{breakurl}
\usepackage[alf,abnt-emphasize=bf]{abntex2cite}	
\usepackage{verbatim}
\usepackage{titlecaps}

%Parâmetros
\setlength{\parindent}{1.3cm}
\setlength{\parskip}{0.2 cm} 


%
\usepackage{hyperref} % controla a formação do índice

\definecolor{blue}{RGB}{0,0,0}
\hypersetup{
     	%pagebackref=true,
		pdftitle={\@title}, 
		pdfauthor={\@author},
    	pdfsubject={\imprimirpreambulo},
	    pdfcreator={LaTeX with abnTeX2},
		pdfkeywords={abnt}{latex}{abntex}{abntex2}{projeto de pesquisa}, 
		colorlinks=true,       		% false: boxed links; true: colored links
    	linkcolor=black,          	% color of internal links
    	citecolor=black,        		% color of links to bibliography
    	filecolor=black,      		% color of file links
		urlcolor=black,
		bookmarksdepth=4}

\newcommand{\largetitle}{\fontsize{14 pt}{0 pt}\selectfont}
\newcommand{\Largetitle}{\fontsize{16 pt}{0 pt}\selectfont}
\renewcommand{\large}{\fontsize{14 pt}{14 pt}\selectfont}
\renewcommand{\Large}{\fontsize{16 pt}{16 pt}\selectfont}
\renewcommand{\ABNTEXsectionfont}{\fontfamily{cmr}\fontseries{b}\selectfont}
\renewcommand{\ABNTEXsectionfontsize}{\normalsize}
\renewcommand{\ABNTEXsubsectionfont}{\fontfamily{cmr}\fontseries{b}\selectfont}
\renewcommand{\ABNTEXsubsectionfontsize}{\normalsize}
\renewcommand{\ABNTEXchapterfontsize}{\Large}

\renewcommand{\afterchapternum}{. }
\renewcommand{\afterchaptertitle}{\par\nobreak \smallskip \hrule \vskip 12pt }
\newcommand{\afterabstract}{\par \nobreak \smallskip \hrule \meskip \afterabstractskip}
%Capa
\renewcommand{\imprimircapa}{%
	\begin{capa}
		\begin{center} 
        \bfseries \Large \imprimirinstituicao \\
        	\vspace{84 pt}
			{\bfseries\large\imprimirtipotrabalho}
			\vspace{42 pt}
          \end{center}
		\begin{center}
			 \bfseries\large  \imprimirtitulo
		\end{center}
                \vspace*{42 pt}
            	\begin{center}
                \bfseries\normalsize Autor:\\
                \bfseries\large \imprimirautor\\
                 \vspace*{28 pt}
				\bfseries\normalsize Orientador:\\
                \bfseries\large\imprimirorientadorRotulo~\imprimirorientador\\
                %\bfseries\normalsize Coorientador:\\
                \bfseries\large \imprimircoorientadorRotulo~\imprimircoorientador \\

 
				\end{center}
			\vfill
            \begin{center}
            	\begin{figure}[h]
                \centering
                \includegraphics[scale=0.15]{Figuras/logoufscar.png}
                \end{figure}
                \vspace{-0.75 cm}
			\imprimirlocal  - \imprimirdata
			\end{center}
				
			\vspace*{1 cm} 
 
	\end{capa}
}

%Folha de rosto
\makeatletter
\renewcommand{\folhaderostocontent}{
	\begin{center} 
        \bfseries \Large \imprimirinstituicao \\
        	\vspace*{70 pt}
			{ \bfseries\large\imprimirtipotrabalho}
			\vspace*{28 pt}
     \end{center}
     \begin{center}
			\bfseries\large\imprimirtitulo 
	\end{center}
    \vspace*{3 cm}
\hspace{.45\textwidth}
\begin{minipage}{.5\textwidth}
	\SingleSpacing
		\imprimirpreambulo
\end{minipage}
\vfill
\imprimirorientadorRotulo~\imprimirorientador
\abntex@ifnotempty{\imprimircoorientador}{
{\imprimircoorientadorRotulo~\imprimircoorientador}}
\vfill
\begin{center}
            	\begin{figure}[h]
                \centering
                \includegraphics[scale=0.15] {Figuras/logoufscar.png}
                \end{figure}
                \vspace{-0.75 cm}
			\imprimirlocal  - \imprimirdata
			\end{center}
				
			\vspace*{1 cm} 
 }
\makeatother




    
 %%Apagar Headeer
\makepagestyle{meuestilo}
  %%cabeçalhos
  \makeevenhead{meuestilo} %%pagina par
     {\thepage}
     {}
     {}
  \makeoddhead{meuestilo} %%pagina ímpar ou com oneside
     {}
     {}
     {\thepage}
  %% rodapé
  \makeevenfoot{meuestilo}
     {} %%pagina par
     {}
     {} 
  \makeoddfoot{meuestilo} %%pagina ímpar ou com oneside
     {}
     {}
     {}

\titulo{ \uppercase{Seu Titulo da Monografia}}
\autor{Seu nome}
\orientador{Seu orientador}
%\coorientador{} 
\local{São Carlos - SP} 
\data{\the\year} 
\instituicao{% 
Universidade Federal de São Carlos - UFSCar \\
Centro de Ciências Exatas e de Tecnologia – CCET \\
Departamento de Engenharia Mecânica – DEMec \\
Curso de Engenharia Mecânica}
\tipotrabalho{Projeto de Monografia}
\preambulo{Modelo canônico de trabalho monográfico acadêmico em conformidade com as normas ABNT apresentado à comunidade de usuários \LaTeX.}


\begin{document}
\pretextual
\imprimircapa 
\newpage
\imprimirfolhaderosto
\newpage

\begin{folhadeaprovacao}
\begin{center}
{\ABNTEXchapterfont\large\imprimirautor}
\vspace*{\fill}\vspace*{\fill}
\begin{center}
\ABNTEXchapterfont\bfseries\Large\imprimirtitulo
\end{center}
\vspace*{\fill}
\hspace{.45\textwidth}
\begin{minipage}{.5\textwidth}
\imprimirpreambulo
\end{minipage}%
\vspace*{\fill}
\end{center}
Trabalho aprovado. \imprimirlocal, \imprimirdata:
\assinatura{\textbf{\imprimirorientadorRotulo~\imprimirorientador} \\ Orientador}
\assinatura{\textbf{Professor} \\ Convidado 1}
\assinatura{\textbf{Professor} \\ Convidado 2}
\assinatura{\textbf{Professor} \\ Convidado 3}
\assinatura{\textbf{Professor} \\ Convidado 4}
\begin{center}
\vspace*{0.5cm}
{\large\imprimirlocal}
\par
{\large\imprimirdata}
\vspace*{1cm}
\end{center}
\end{folhadeaprovacao}


\begin{dedicatoria}
	\vspace*{\fill}
    \hspace*{\fill}
    		Insira aqui a sua dedicatória ....
    \hspace*{\fill}
	\vspace*{\fill}
\end{dedicatoria}

\begin{agradecimentos}
	Insira aqui os seus agradecimentos ...
\end{agradecimentos}


\begin{epigrafe}
	\vspace*{\fill}
		\begin{flushright}
			\textit{‘‘Frase famosa que eu gosto"\\
            (Autor,Livro,etc..)}
		\end{flushright}
\end{epigrafe}


Inserir aqui o seu resumo para o trabalho

\textbf{Palavras-chave}: Palavras. Chaves. Aqui.
\listoffigures*
\cleardoublepage
\listoftables*
\cleardoublepage
\input{Lista_de_Siglas_Simbolos}
\tableofcontents*
\cleardoublepage
\begin{resumo}
\vspace{-0.42 cm}
\par \nobreak \hrule 
\vspace{12 pt}
\noindent
Inserir aqui o seu resumo para o trabalho

\textbf{Palavras-chave}: Palavras. Chaves. Aqui.
\end{resumo}
\clearpage


\textual %Aqui começa a parte com páginas numeradas
%\pagestyle{meuestilo}
\chapter{Introdução}
%Crie arquivos a parte e use o comando \input{} para inserir o seu texto no capitulo.


\postextual
%O arquivo de bibliografi precisa estar em documento a parte, como você vai ver em todos os tutoriais que usar na net
%% IMPORTANTE: remova o % das duas linhas abaixo para que a bibliografia funcione
%\bibliographystyle{abntex2-alf} 
%\bibliography{Bib.bib}

\end{document}
